% Options for packages loaded elsewhere
% Options for packages loaded elsewhere
\PassOptionsToPackage{unicode}{hyperref}
\PassOptionsToPackage{hyphens}{url}
%
\documentclass[
  english,
  russian,
  12pt,
  a4paper,
  DIV=11,
  numbers=noendperiod]{scrreprt}
\usepackage{xcolor}
\usepackage{amsmath,amssymb}
\setcounter{secnumdepth}{5}
\usepackage{iftex}
\ifPDFTeX
  \usepackage[T1]{fontenc}
  \usepackage[utf8]{inputenc}
  \usepackage{textcomp} % provide euro and other symbols
\else % if luatex or xetex
  \usepackage{unicode-math} % this also loads fontspec
  \defaultfontfeatures{Scale=MatchLowercase}
  \defaultfontfeatures[\rmfamily]{Ligatures=TeX,Scale=1}
\fi
\usepackage{lmodern}
\ifPDFTeX\else
  % xetex/luatex font selection
\fi
% Use upquote if available, for straight quotes in verbatim environments
\IfFileExists{upquote.sty}{\usepackage{upquote}}{}
\IfFileExists{microtype.sty}{% use microtype if available
  \usepackage[]{microtype}
  \UseMicrotypeSet[protrusion]{basicmath} % disable protrusion for tt fonts
}{}
\usepackage{setspace}
% Make \paragraph and \subparagraph free-standing
\makeatletter
\ifx\paragraph\undefined\else
  \let\oldparagraph\paragraph
  \renewcommand{\paragraph}{
    \@ifstar
      \xxxParagraphStar
      \xxxParagraphNoStar
  }
  \newcommand{\xxxParagraphStar}[1]{\oldparagraph*{#1}\mbox{}}
  \newcommand{\xxxParagraphNoStar}[1]{\oldparagraph{#1}\mbox{}}
\fi
\ifx\subparagraph\undefined\else
  \let\oldsubparagraph\subparagraph
  \renewcommand{\subparagraph}{
    \@ifstar
      \xxxSubParagraphStar
      \xxxSubParagraphNoStar
  }
  \newcommand{\xxxSubParagraphStar}[1]{\oldsubparagraph*{#1}\mbox{}}
  \newcommand{\xxxSubParagraphNoStar}[1]{\oldsubparagraph{#1}\mbox{}}
\fi
\makeatother


\usepackage{longtable,booktabs,array}
\usepackage{calc} % for calculating minipage widths
% Correct order of tables after \paragraph or \subparagraph
\usepackage{etoolbox}
\makeatletter
\patchcmd\longtable{\par}{\if@noskipsec\mbox{}\fi\par}{}{}
\makeatother
% Allow footnotes in longtable head/foot
\IfFileExists{footnotehyper.sty}{\usepackage{footnotehyper}}{\usepackage{footnote}}
\makesavenoteenv{longtable}
\usepackage{graphicx}
\makeatletter
\newsavebox\pandoc@box
\newcommand*\pandocbounded[1]{% scales image to fit in text height/width
  \sbox\pandoc@box{#1}%
  \Gscale@div\@tempa{\textheight}{\dimexpr\ht\pandoc@box+\dp\pandoc@box\relax}%
  \Gscale@div\@tempb{\linewidth}{\wd\pandoc@box}%
  \ifdim\@tempb\p@<\@tempa\p@\let\@tempa\@tempb\fi% select the smaller of both
  \ifdim\@tempa\p@<\p@\scalebox{\@tempa}{\usebox\pandoc@box}%
  \else\usebox{\pandoc@box}%
  \fi%
}
% Set default figure placement to htbp
\def\fps@figure{htbp}
\makeatother



\ifLuaTeX
\usepackage[bidi=basic,provide=*]{babel}
\else
\usepackage[bidi=default,provide=*]{babel}
\fi
% get rid of language-specific shorthands (see #6817):
\let\LanguageShortHands\languageshorthands
\def\languageshorthands#1{}


\setlength{\emergencystretch}{3em} % prevent overfull lines

\providecommand{\tightlist}{%
  \setlength{\itemsep}{0pt}\setlength{\parskip}{0pt}}



 
\usepackage[style=gost-numeric,backend=biber,langhook=extras,autolang=other*]{biblatex}
\addbibresource{bib/cite.bib}

\usepackage[]{csquotes}

\KOMAoption{captions}{tableheading}
\usepackage{indentfirst}
\usepackage{float}
\floatplacement{figure}{H}
\usepackage{libertine}
\usepackage{indentfirst}
\usepackage{float}
\floatplacement{figure}{H}
\usepackage[math,RM={Scale=0.94},SS={Scale=0.94},SScon={Scale=0.94},TT={Scale=MatchLowercase,FakeStretch=0.9},DefaultFeatures={Ligatures=Common}]{plex-otf}
\makeatletter
\@ifpackageloaded{caption}{}{\usepackage{caption}}
\AtBeginDocument{%
\ifdefined\contentsname
  \renewcommand*\contentsname{Содержание}
\else
  \newcommand\contentsname{Содержание}
\fi
\ifdefined\listfigurename
  \renewcommand*\listfigurename{Список иллюстраций}
\else
  \newcommand\listfigurename{Список иллюстраций}
\fi
\ifdefined\listtablename
  \renewcommand*\listtablename{Список таблиц}
\else
  \newcommand\listtablename{Список таблиц}
\fi
\ifdefined\figurename
  \renewcommand*\figurename{Рисунок}
\else
  \newcommand\figurename{Рисунок}
\fi
\ifdefined\tablename
  \renewcommand*\tablename{Таблица}
\else
  \newcommand\tablename{Таблица}
\fi
}
\@ifpackageloaded{float}{}{\usepackage{float}}
\floatstyle{ruled}
\@ifundefined{c@chapter}{\newfloat{codelisting}{h}{lop}}{\newfloat{codelisting}{h}{lop}[chapter]}
\floatname{codelisting}{Список}
\newcommand*\listoflistings{\listof{codelisting}{Листинги}}
\makeatother
\makeatletter
\makeatother
\makeatletter
\@ifpackageloaded{caption}{}{\usepackage{caption}}
\@ifpackageloaded{subcaption}{}{\usepackage{subcaption}}
\makeatother
\usepackage{bookmark}
\IfFileExists{xurl.sty}{\usepackage{xurl}}{} % add URL line breaks if available
\urlstyle{same}
\hypersetup{
  pdftitle={Лабораторная работа №10},
  pdfauthor={Перфилов Александр Константинович \textbar{} группа НПИбд 03-24},
  pdflang={ru-RU},
  hidelinks,
  pdfcreator={LaTeX via pandoc}}


\title{Лабораторная работа №10}
\usepackage{etoolbox}
\makeatletter
\providecommand{\subtitle}[1]{% add subtitle to \maketitle
  \apptocmd{\@title}{\par {\large #1 \par}}{}{}
}
\makeatother
\subtitle{Основы работы с модулями ядра операционной системы}
\author{Перфилов Александр Константинович \textbar{} группа НПИбд 03-24}
\date{}
\begin{document}
\maketitle

\renewcommand*\contentsname{Содержание}
{
\setcounter{tocdepth}{1}
\tableofcontents
}
\listoffigures
\listoftables

\setstretch{1.5}
\chapter{Цель
работы}\label{ux446ux435ux43bux44c-ux440ux430ux431ux43eux442ux44b}

Получить навыки работы с утилитами управления модулями ядра операционной
системы.

\chapter{Выполнение лабораторной
работы}\label{ux432ux44bux43fux43eux43bux43dux435ux43dux438ux435-ux43bux430ux431ux43eux440ux430ux442ux43eux440ux43dux43eux439-ux440ux430ux431ux43eux442ux44b}

\section{Управление модулями ядра из командной
строки}\label{ux443ux43fux440ux430ux432ux43bux435ux43dux438ux435-ux43cux43eux434ux443ux43bux44fux43cux438-ux44fux434ux440ux430-ux438ux437-ux43aux43eux43cux430ux43dux434ux43dux43eux439-ux441ux442ux440ux43eux43aux438}

Посмотрим, какие устройства имеются в системе и какие модули ядра с ними
связаны (рис. \autocite*{fig:001}).

\begin{figure}

{\centering \includegraphics[width=0.7\linewidth,height=\textheight,keepaspectratio]{image/1.jpg}

}

\caption{lspci -k}

\end{figure}%

Каждая строка в выводе содержит следующую информацию:

\begin{enumerate}
\def\labelenumi{\arabic{enumi}.}
\item
  Идентификатор устройства (например, 0:00.0): Уникальный адрес
  устройства на шине PCI.
\item
  Тип устройства (например, Host bridge, VGA compatible controller):
  Описание типа устройства.
\item
  Производитель и модель (например, Intel Corporation 440FX - 82441FX
  PMC {[}Natoma{]}): Информация о производителе и модели устройства.
\item
  Версия (например, (rev 02)): Версия устройства.
\item
  Драйвер ядра (например, Kernel driver in use: ata\_piix): Драйвер,
  который в данный момент используется для управления устройством.
\item
  Модули ядра (например, Kernel modules: ata\_piix, ata\_generic):
  Модули ядра, которые могут быть загружены для работы с данным
  устройством.
\end{enumerate}

\textbf{Примеры устройств}

\begin{enumerate}
\def\labelenumi{\arabic{enumi}.}
\item
  Host bridge: Устройство, которое соединяет процессор с другими
  компонентами системы.
\item
  IDE interface: Устройство для управления IDE-накопителями. Использует
  драйвер ata\_piix.
\item
  VGA compatible controller: Видеоконтроллер, который управляет
  графикой. Использует драйвер vmwgfx.
\item
  Ethernet controller: Сетевой контроллер для подключения к сети.
  Использует драйвер e1000.
\item
  Multimedia audio controller: Звуковой контроллер для обработки
  аудиосигналов. Использует драйвер snd\_intel8x0.
\item
  USB controller: Контроллер для управления USB-устройствами. Использует
  драйвер ohci-pci и ehci-pci для разных USB-портов.
\item
  SATA controller: Контроллер для управления SATA-накопителями.
  Использует драйвер ahci.
\end{enumerate}

Посмотрим, какие модули ядра загружены. Посмотрим, загружен ли модуль
ext4 (нет). (рис. \autocite*{fig:002}).

\begin{figure}

{\centering \includegraphics[width=0.7\linewidth,height=\textheight,keepaspectratio]{image/2.jpg}

}

\caption{загруженные модули}

\end{figure}%

Загрузим модуль ядра ext4. Убедимся, что модуль загружен. Посмотрим
информацию о модуле ядра ext4(рис. \autocite*{fig:003}).

\begin{figure}

{\centering \includegraphics[width=0.7\linewidth,height=\textheight,keepaspectratio]{image/3.jpg}

}

\caption{загрузка ядра ext4 и информация}

\end{figure}%

\begin{enumerate}
\def\labelenumi{\arabic{enumi}.}
\item
  filename:

  \begin{itemize}
  \item
    /lib/modules/5.14.0-427.35.1.el9\_4.x86\_64/kernel/fs/ext4/ext4.ko.xz
  \item
    Указывает путь к файлу модуля ядра (в данном случае, сжатый файл
    ext4.ko).
  \end{itemize}
\item
  description:

  \begin{itemize}
  \item
    Fourth Extended Filesystem
  \item
    Краткое описание модуля.
  \end{itemize}
\item
  author:

  \begin{itemize}
  \tightlist
  \item
    Указывает авторов разработки модуля (например, Remy Card, Stephen
    Tweedie и др.).
  \end{itemize}
\item
  license:

  \begin{itemize}
  \item
    GPL
  \item
    Указывает лицензию, под которой распространяется модуль (в данном
    случае, GNU General Public License).
  \end{itemize}
\item
  depends:

  \begin{itemize}
  \item
    mbcache, jbd2
  \item
    Указывает зависимости модуля от других модулей ядра. Этот модуль
    зависит от mbcache и jbd2.
  \end{itemize}
\item
  alias:

  \begin{itemize}
  \item
    fs-ext4, ext3, fs-ext3, ext2, fs-ext2
  \item
    Указывает альтернативные имена для данного модуля, что позволяет
    системе загружать модуль по другим именам.
  \end{itemize}
\item
  rhelversion:

  \begin{itemize}
  \item
    9.4
  \item
    Указывает на версию Red Hat Enterprise Linux, с которой этот модуль
    совместим.
  \end{itemize}
\item
  srcversion:

  \begin{itemize}
  \item
    48ACD3511F499E70E80D5E4
  \item
    Уникальный идентификатор версии исходного кода модуля.
  \end{itemize}
\item
  vermagic:

  \begin{itemize}
  \item
    5.14.0-427.35.1.el9\_4.x86\_64 SMP preempt mod\_unload modversions
  \item
    Указывает на версию ядра, для которой был скомпилирован модуль, а
    также на параметры конфигурации (например, поддержка SMP,
    прерываемости и т.д.).
  \end{itemize}
\item
  signature:

  \begin{itemize}
  \item
    Содержит информацию о цифровой подписи модуля, включая алгоритм
    хеширования (sha256) и саму подпись.
  \item
    Это обеспечивает безопасность и целостность модуля.
  \end{itemize}
\end{enumerate}

Попробуем выгрузить модуль ядра ext4 (рис. \autocite*{fig:004}).

Система сообщает, что модуль нельзя выгрузить так как он используется.

\begin{figure}

{\centering \includegraphics[width=0.7\linewidth,height=\textheight,keepaspectratio]{image/4.jpg}

}

\caption{выгрузка модулей}

\end{figure}%

\section{Загрузка модулей ядра с
параметрами}\label{ux437ux430ux433ux440ux443ux437ux43aux430-ux43cux43eux434ux443ux43bux435ux439-ux44fux434ux440ux430-ux441-ux43fux430ux440ux430ux43cux435ux442ux440ux430ux43cux438}

Посмотрим, загружен ли модуль bluetooth. hагрузим модуль ядра bluetooth.
Посмотрим список модулей ядра, отвечающих за работу с Bluetooth.
посмотрим информацию о модуле bluetooth. Выгрузим модуль ядра bluetooth
(рис. \autocite*{fig:005}).

\begin{figure}

{\centering \includegraphics[width=0.7\linewidth,height=\textheight,keepaspectratio]{image/5.jpg}

}

\caption{модуль bluetooth}

\end{figure}%

\textbf{параметры модуля Bluetooth}

\begin{enumerate}
\def\labelenumi{\arabic{enumi}.}
\item
  disable\_esco:

  \begin{itemize}
  \item
    Описание: Отключает создание соединений eSCO (Extended Synchronous
    Connection-Oriented).
  \item
    Тип: bool (логический, true/false)
  \end{itemize}
\item
  disable\_ertm:

  \begin{itemize}
  \item
    Описание: Отключает режим улучшенной повторной передачи (Enhanced
    Retransmission Mode).
  \item
    Тип: bool (логический, true/false)
  \end{itemize}
\item
  enable\_ecred:

  \begin{itemize}
  \item
    Описание: Включает режим улучшенного управления потоком (Enhanced
    Credit Flow Control).
  \item
    Тип: bool (логический, true/false)
  \end{itemize}
\end{enumerate}

\section{Обновление ядра
системы}\label{ux43eux431ux43dux43eux432ux43bux435ux43dux438ux435-ux44fux434ux440ux430-ux441ux438ux441ux442ux435ux43cux44b}

Посмотрим версию ядра, используемую в операционной системе. Выведем на
экран список пакетов, относящихся к ядру операционной системы. Обновим
систему, чтобы убедиться, что все существующие пакеты обновлены(рис.
\autocite*{fig:006})

\begin{figure}

{\centering \includegraphics[width=0.7\linewidth,height=\textheight,keepaspectratio]{image/6.jpg}

}

\caption{версия ядра}

\end{figure}%

Обновим ядро операционной системы, а затем саму операционную систему
(рис. \autocite*{fig:007})

\begin{figure}

{\centering \includegraphics[width=0.7\linewidth,height=\textheight,keepaspectratio]{image/7.jpg}

}

\caption{обновим ядро и систему}

\end{figure}%

Посмотрим версию ядра, используемую в операционной системе (выбрана
последняя версия) (рис. \autocite*{fig:008}).

\begin{figure}

{\centering \includegraphics[width=0.7\linewidth,height=\textheight,keepaspectratio]{image/8.jpg}

}

\caption{версия ОС}

\end{figure}%

\chapter{Контрольные
вопросы}\label{ux43aux43eux43dux442ux440ux43eux43bux44cux43dux44bux435-ux432ux43eux43fux440ux43eux441ux44b}

\begin{enumerate}
\def\labelenumi{\arabic{enumi}.}
\item
  Какой командой показать текущую версию ядра?

  \textbf{uname -r}
\item
  Как посмотреть более подробную информацию о текущей версии ядра?

  \textbf{uname -a}

  Эта команда покажет полную информацию о системе, включая версию ядра.
\item
  Какая команда показывает список загруженных модулей ядра?

  \textbf{lsmod}
\item
  Как определить параметры модуля ядра?

  \textbf{modinfo }

  Замените на имя интересующего вас модуля.
\item
  Как выгрузить модуль ядра?

  \textbf{rmmod }

  Или можно использовать:
\end{enumerate}

\textbf{modprobe -r }

\begin{enumerate}
\def\labelenumi{\arabic{enumi}.}
\setcounter{enumi}{5}
\item
  Что делать, если вы получили сообщение об ошибке при попытке выгрузить
  модуль ядра?

  \begin{itemize}
  \item
    Убедитесь, что модуль не используется другими процессами.
    Используйте команду lsof или fuser, чтобы найти процессы,
    использующие модуль.
  \item
    Если модуль является зависимостью для других модулей, сначала нужно
    выгрузить их.
  \item
    Попробуйте использовать modprobe -r вместо rmmod, так как он
    автоматически обработает зависимости.
  \end{itemize}
\item
  Как определить, какие параметры модуля ядра поддерживаются?

  \textbf{modinfo -p }
\item
  Как установить новую версию ядра?

  \begin{itemize}
  \item
    После установки перезагрузите систему:

    \textbf{reboot}
  \item
    Выберите новую версию ядра в меню загрузчика (GRUB), если это
    необходимо.
  \end{itemize}
\end{enumerate}

\chapter{Вывод}\label{ux432ux44bux432ux43eux434}

В ходе выполнения лабораторной работы я получил навыки работы с
утилитами управления модулями ядра операционной системы.


\printbibliography



\end{document}
