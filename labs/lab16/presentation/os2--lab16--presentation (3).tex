% Options for packages loaded elsewhere
% Options for packages loaded elsewhere
\PassOptionsToPackage{unicode}{hyperref}
\PassOptionsToPackage{hyphens}{url}
%
\documentclass[
  ignorenonframetext,
  aspectratio=169,
  russian,
]{beamer}
\newif\ifbibliography
\usepackage{pgfpages}
\setbeamertemplate{caption}[numbered]
\setbeamertemplate{caption label separator}{: }
\setbeamercolor{caption name}{fg=normal text.fg}
\beamertemplatenavigationsymbolshorizontal
% Prevent slide breaks in the middle of a paragraph
\widowpenalties 1 10000
\raggedbottom
\AtBeginPart{
  \frame{\partpage}
}
\AtBeginSection{
  \ifbibliography
  \else
    \frame{\sectionpage}
  \fi
}
\AtBeginSubsection{
  \frame{\subsectionpage}
}
\usepackage{iftex}
\ifPDFTeX
  \usepackage[T1]{fontenc}
  \usepackage[utf8]{inputenc}
  \usepackage{textcomp} % provide euro and other symbols
\else % if luatex or xetex
  \usepackage{unicode-math} % this also loads fontspec
  \defaultfontfeatures{Scale=MatchLowercase}
  \defaultfontfeatures[\rmfamily]{Ligatures=TeX,Scale=1}
\fi
\usepackage{lmodern}

\usetheme[]{JuanLesPins}
\usecolortheme[]{beaver}
\usefonttheme[]{professionalfonts}
\usefonttheme{serif} % use mainfont rather than sansfont for slide text
\useinnertheme[]{rounded}
\useoutertheme[]{infolines}
\ifPDFTeX\else
  % xetex/luatex font selection
  \setmainfont[Ligatures=TeX]{Liberation Serif}
  \setsansfont[Ligatures=TeX,Scale=MatchLowercase]{Liberation Sans}
  \setmonofont[Scale=MatchLowercase,Scale=0.9]{Liberation Mono}
\fi
% Use upquote if available, for straight quotes in verbatim environments
\IfFileExists{upquote.sty}{\usepackage{upquote}}{}
\IfFileExists{microtype.sty}{% use microtype if available
  \usepackage[]{microtype}
  \UseMicrotypeSet[protrusion]{basicmath} % disable protrusion for tt fonts
}{}


\usepackage{longtable,booktabs,array}
\usepackage{calc} % for calculating minipage widths
\usepackage{caption}
% Make caption package work with longtable
\makeatletter
\def\fnum@table{\tablename~\thetable}
\makeatother
\usepackage{graphicx}
\makeatletter
\newsavebox\pandoc@box
\newcommand*\pandocbounded[1]{% scales image to fit in text height/width
  \sbox\pandoc@box{#1}%
  \Gscale@div\@tempa{\textheight}{\dimexpr\ht\pandoc@box+\dp\pandoc@box\relax}%
  \Gscale@div\@tempb{\linewidth}{\wd\pandoc@box}%
  \ifdim\@tempb\p@<\@tempa\p@\let\@tempa\@tempb\fi% select the smaller of both
  \ifdim\@tempa\p@<\p@\scalebox{\@tempa}{\usebox\pandoc@box}%
  \else\usebox{\pandoc@box}%
  \fi%
}
% Set default figure placement to htbp
\def\fps@figure{htbp}
\makeatother



\ifLuaTeX
\usepackage[bidi=basic,provide=*]{babel}
\else
\usepackage[bidi=default,provide=*]{babel}
\fi
\ifPDFTeX
\else
\babelfont{rm}[Ligatures=TeX]{Liberation Serif}
\fi
% get rid of language-specific shorthands (see #6817):
\let\LanguageShortHands\languageshorthands
\def\languageshorthands#1{}


\setlength{\emergencystretch}{3em} % prevent overfull lines

\providecommand{\tightlist}{%
  \setlength{\itemsep}{0pt}\setlength{\parskip}{0pt}}



 

\usepackage[]{csquotes}

\usepackage{libertine}
\makeatletter
\@ifpackageloaded{caption}{}{\usepackage{caption}}
\AtBeginDocument{%
\ifdefined\contentsname
  \renewcommand*\contentsname{Содержание}
\else
  \newcommand\contentsname{Содержание}
\fi
\ifdefined\listfigurename
  \renewcommand*\listfigurename{Список иллюстраций}
\else
  \newcommand\listfigurename{Список иллюстраций}
\fi
\ifdefined\listtablename
  \renewcommand*\listtablename{Список таблиц}
\else
  \newcommand\listtablename{Список таблиц}
\fi
\ifdefined\figurename
  \renewcommand*\figurename{Рисунок}
\else
  \newcommand\figurename{Рисунок}
\fi
\ifdefined\tablename
  \renewcommand*\tablename{Таблица}
\else
  \newcommand\tablename{Таблица}
\fi
}
\@ifpackageloaded{float}{}{\usepackage{float}}
\floatstyle{ruled}
\@ifundefined{c@chapter}{\newfloat{codelisting}{h}{lop}}{\newfloat{codelisting}{h}{lop}[chapter]}
\floatname{codelisting}{Список}
\newcommand*\listoflistings{\listof{codelisting}{Листинги}}
\makeatother
\makeatletter
\makeatother
\makeatletter
\@ifpackageloaded{caption}{}{\usepackage{caption}}
\@ifpackageloaded{subcaption}{}{\usepackage{subcaption}}
\makeatother

\usepackage{bookmark}
\IfFileExists{xurl.sty}{\usepackage{xurl}}{} % add URL line breaks if available
\urlstyle{same}
\hypersetup{
  pdftitle={Лабораторная работа №16},
  pdfauthor={Перфилов Александр Константинович \textbar{} Группа НПИбд-03-24},
  pdflang={ru-RU},
  hidelinks,
  pdfcreator={LaTeX via pandoc}}


\title{Лабораторная работа №16}
\subtitle{Програмный RAID}
\author{Перфилов Александр Константинович \textbar{} Группа НПИбд-03-24}
\date{Invalid Date}
\institute{Российский университет дружбы народов, Москва, Россия}

\begin{document}
\frame{\titlepage}


\begin{frame}{1. Информация}
\phantomsection\label{ux438ux43dux444ux43eux440ux43cux430ux446ux438ux44f}
\begin{block}{1.1 Докладчик}
\phantomsection\label{ux434ux43eux43aux43bux430ux434ux447ux438ux43a}
\begin{columns}[c]
\begin{column}{0.7\linewidth}
\begin{itemize}[<+->]
\tightlist
\item
  Перфилов Александр Константинович
\item
  Группа НПИбд-03-24
\item
  Российский университет дружбы народов
\item
  \url{https://github.com/AlexanderPErfilovKonstantinivich?tab=repositories}
\end{itemize}
\end{column}

\begin{column}{0.3\linewidth}
\end{column}
\end{columns}
\end{block}
\end{frame}

\begin{frame}{2. Цель работы}
\phantomsection\label{ux446ux435ux43bux44c-ux440ux430ux431ux43eux442ux44b}
Освоить работу с RAID-массивами при помощи утилиты mdadm.
\end{frame}

\begin{frame}{3. Задание}
\phantomsection\label{ux437ux430ux434ux430ux43dux438ux435}
\begin{enumerate}[<+->]
\tightlist
\item
  Прочитайте руководство по работе с утилитами fdisk, sfdisk и mdadm.
\item
  Добавить три диска на виртуальную машину (объёмом от 512 MiB каждый).
  При помощи sfdisk создать на каждом из дисков по одной партиции, задав
  тип раздела для RAID.
\item
  Создать массив RAID 1 из двух дисков, смонтировать его. Эмитировать
  сбой одного из дисков массива, удалить искусственно выведенный из
  строя диск, добавить в массив работающий диск.
\item
  Создать массив RAID 1 из двух дисков, смонтировать его. Добавить к
  массиву третий диск. Эмитировать сбой одного из дисков массива.
  Проанализировать состояние массива, указать различия по сравнению с
  предыдущим случаем.
\item
  Создать массив RAID 1 из двух дисков, смонтировать его. Добавить к
  массиву третий диск. Изменить тип массива с RAID1 на RAID5, изменить
  число дисков в массиве с 2 на 3. Проанализировать состояние массива,
  указать различия по сравнению с предыдущим случаем.
\end{enumerate}
\end{frame}

\begin{frame}[fragile]{4. Выполнение лабораторной работы}
\phantomsection\label{ux432ux44bux43fux43eux43bux43dux435ux43dux438ux435-ux43bux430ux431ux43eux440ux430ux442ux43eux440ux43dux43eux439-ux440ux430ux431ux43eux442ux44b}
\begin{block}{4.1 Создание виртуальных носителей}
\phantomsection\label{ux441ux43eux437ux434ux430ux43dux438ux435-ux432ux438ux440ux442ux443ux430ux43bux44cux43dux44bux445-ux43dux43eux441ux438ux442ux435ux43bux435ux439}
Я добавила к своей виртуальной машине три диска размером 512 MiB к
контроллеру SATA.

\begin{figure}

{\centering \includegraphics[width=0.7\linewidth,height=\textheight,keepaspectratio]{./home/akpperfilov/work/study/2025-2026/Основы администрирования операционных систем/os2/study_2025-2026_os2/labs/lab16/presentation/image/1.jpg}

}

\caption{Создание дисков}

\end{figure}%
\end{block}

\begin{block}{4.2 Создание RAID-диска}
\phantomsection\label{ux441ux43eux437ux434ux430ux43dux438ux435-raid-ux434ux438ux441ux43aux430}
\begin{enumerate}[<+->]
\tightlist
\item
  Проверила наличие созданных дисков, введя команду:
  \texttt{fdisk\ -l\ \textbar{}\ grep\ /dev/sd} Если предыдущая работа
  по LVM была выполнена успешно, то в системе я увидела добавленные
  диски, отображающиеся как /dev/sdd, /dev/sde, /dev/sdf.
\end{enumerate}

\begin{figure}

{\centering \includegraphics[width=0.7\linewidth,height=\textheight,keepaspectratio]{./home/akpperfilov/work/study/2025-2026/Основы администрирования операционных систем/os2/study_2025-2026_os2/labs/lab16/presentation/image/2.jpg}

}

\caption{Проверка наличия дисков}

\end{figure}%

\begin{enumerate}[<+->]
\setcounter{enumi}{1}
\tightlist
\item
  Я создала раздел на каждом из дисков.
\end{enumerate}

\begin{figure}

{\centering \includegraphics[width=0.7\linewidth,height=\textheight,keepaspectratio]{./home/akpperfilov/work/study/2025-2026/Основы администрирования операционных систем/os2/study_2025-2026_os2/labs/lab16/presentation/image/3.jpg}

}

\caption{Создание разделов}

\end{figure}%

\begin{figure}

{\centering \includegraphics[width=0.7\linewidth,height=\textheight,keepaspectratio]{./home/akpperfilov/work/study/2025-2026/Основы администрирования операционных систем/os2/study_2025-2026_os2/labs/lab16/presentation/image/4.jpg}

}

\caption{Создание разделов}

\end{figure}%

\begin{figure}

{\centering \includegraphics[width=0.7\linewidth,height=\textheight,keepaspectratio]{./home/akpperfilov/work/study/2025-2026/Основы администрирования операционных систем/os2/study_2025-2026_os2/labs/lab16/presentation/image/5.jpg}

}

\caption{Создание разделов}

\end{figure}%

\begin{enumerate}[<+->]
\setcounter{enumi}{2}
\tightlist
\item
  Я проверила текущий тип созданных разделов с помощью команд:
  \texttt{sfdisk\ -\/-print-id\ /dev/sdd\ 1\ sfdisk\ -\/-print-id\ /dev/sde\ 1\ sfdisk\ -\/-print-id\ /dev/sdf\ 1}
  Все созданные мной разделы имеют тип Linux.
\end{enumerate}

\begin{figure}

{\centering \includegraphics[width=0.7\linewidth,height=\textheight,keepaspectratio]{./home/akpperfilov/work/study/2025-2026/Основы администрирования операционных систем/os2/study_2025-2026_os2/labs/lab16/presentation/image/6.jpg}

}

\caption{Текущий тип разделов}

\end{figure}%

\begin{enumerate}[<+->]
\setcounter{enumi}{3}
\tightlist
\item
  Я просматривала, какие типы партиций, относящиеся к RAID, можно
  задать, использовав команду:
  \texttt{sfdisk\ -T\ \textbar{}\ grep\ -i\ raid}.
\end{enumerate}

\begin{figure}

{\centering \includegraphics[width=0.7\linewidth,height=\textheight,keepaspectratio]{./home/akpperfilov/work/study/2025-2026/Основы администрирования операционных систем/os2/study_2025-2026_os2/labs/lab16/presentation/image/7.jpg}

}

\caption{Типы партиций}

\end{figure}%

\begin{enumerate}[<+->]
\setcounter{enumi}{4}
\tightlist
\item
  Затем я установила тип разделов в Linux raid autodetect следующими
  командами:
  \texttt{sfdisk\ -\/-change-id\ /dev/sdd\ 1\ fd\ sfdisk\ -\/-change-id\ /dev/sde\ 1\ fd\ sfdisk\ -\/-change-id\ /dev/sdf\ 1\ fd}.
\end{enumerate}

\begin{figure}

{\centering \includegraphics[width=0.7\linewidth,height=\textheight,keepaspectratio]{./home/akpperfilov/work/study/2025-2026/Основы администрирования операционных систем/os2/study_2025-2026_os2/labs/lab16/presentation/image/8.jpg}

}

\caption{Установка типа разделов}

\end{figure}%

\begin{enumerate}[<+->]
\setcounter{enumi}{5}
\tightlist
\item
  Я проверила состояние дисков с помощью команд:
  \texttt{sfdisk\ -l\ /dev/sdd\ sfdisk\ -l\ /dev/sde\ sfdisk\ -l\ /dev/sdf}.
\end{enumerate}

\begin{figure}

{\centering \includegraphics[width=0.7\linewidth,height=\textheight,keepaspectratio]{./home/akpperfilov/work/study/2025-2026/Основы администрирования операционных систем/os2/study_2025-2026_os2/labs/lab16/presentation/image/9.jpg}

}

\caption{Проверка состояния дисков}

\end{figure}%

\begin{enumerate}[<+->]
\setcounter{enumi}{6}
\tightlist
\item
  Используя утилиту mdadm, я создала массив RAID 1 из двух дисков с
  помощью команды:
  \texttt{mdadm\ -\/-create\ -\/-verbose\ /dev/md0\ -\/-level=1\ -\/-raid-devices=2\ /dev/sdd1\ /dev/sde1}
  (рис. {[}-@fig:019{]}).
\end{enumerate}

\begin{figure}

{\centering \includegraphics[width=0.7\linewidth,height=\textheight,keepaspectratio]{./home/akpperfilov/work/study/2025-2026/Основы администрирования операционных систем/os2/study_2025-2026_os2/labs/lab16/presentation/image/10.jpg}

}

\caption{Создание массива}

\end{figure}%

\begin{enumerate}[<+->]
\setcounter{enumi}{7}
\tightlist
\item
  Я проверила состояние массива RAID, используя следующие команды:
  \texttt{cat\ /proc/mdstat}, \texttt{mdadm\ -\/-query\ /dev/md0},
  \texttt{mdadm\ -\/-detail\ /dev/md0}.
\end{enumerate}

\begin{figure}

{\centering \includegraphics[width=0.7\linewidth,height=\textheight,keepaspectratio]{./home/akpperfilov/work/study/2025-2026/Основы администрирования операционных систем/os2/study_2025-2026_os2/labs/lab16/presentation/image/11.jpg}

}

\caption{проверка состояния массива}

\end{figure}%

\begin{enumerate}[<+->]
\setcounter{enumi}{8}
\tightlist
\item
  Я создала файловую систему на RAID с помощью команды:
  \texttt{mkfs.ext4\ /dev/md0}.
\end{enumerate}

\begin{figure}

{\centering \includegraphics[width=0.7\linewidth,height=\textheight,keepaspectratio]{./home/akpperfilov/work/study/2025-2026/Основы администрирования операционных систем/os2/study_2025-2026_os2/labs/lab16/presentation/image/12.jpg}

}

\caption{Создание файловой системы}

\end{figure}%

\begin{enumerate}[<+->]
\setcounter{enumi}{9}
\tightlist
\item
  Я подмонтировала RAID, создав каталог: \texttt{mkdir\ /data},
  \texttt{mount\ /dev/md0\ /data}.
\end{enumerate}

\begin{figure}

{\centering \includegraphics[width=0.7\linewidth,height=\textheight,keepaspectratio]{./home/akpperfilov/work/study/2025-2026/Основы администрирования операционных систем/os2/study_2025-2026_os2/labs/lab16/presentation/image/13.jpg}

}

\caption{Монтировка Raid}

\end{figure}%
\end{block}
\end{frame}

\begin{frame}{5. Ответы на контрольные вопросы}
\phantomsection\label{ux43eux442ux432ux435ux442ux44b-ux43dux430-ux43aux43eux43dux442ux440ux43eux43bux44cux43dux44bux435-ux432ux43eux43fux440ux43eux441ux44b}
\begin{enumerate}[<+->]
\item
  RAID (Redundant Array of Independent Disks) -- это технология, которая
  позволяет объединять несколько физических дисков в одну логическую
  единицу для повышения производительности, надежности и/или
  резервирования данных. RAID позволяет распределять данные между
  дисками, что обеспечивает защиту от потери данных при отказе одного
  или нескольких дисков, а также может увеличить скорость чтения и
  записи.
\item
  Типы RAID-массивов Существует несколько уровней RAID, каждый из
  которых обеспечивает свои особенности в области производительности,
  резервирования данных и структуры. Основные типы RAID-массивов:
\end{enumerate}

RAID 0 RAID 1 RAID 5 RAID 6 RAID 10 (1+0) RAID 2 RAID 3 RAID 4 RAID 50
(сочетание RAID 5 и RAID 0) RAID 60 (сочетание RAID 6 и RAID 0)

\begin{enumerate}[<+->]
\setcounter{enumi}{2}
\tightlist
\item
  Описание уровня RAID
\end{enumerate}

RAID 0

Алгоритм работы

Данные разбиваются на блоки и параллельно распределяются по всем дискам
в массиве. Это обеспечит высокую скорость записи и чтения. Назначение

RAID 0 обеспечивает максимальную производительность, но не содержит
резервирования данных. В случае отказа одного из дисков все данные
теряются.

Примеры применения

RAID 0 чаще всего используется в системах, требующих высокой скорости
обработки данных, например в игровой индустрии, видеоредакторах и других
приложениях, где важна высокая производительность.

RAID 1

Алгоритм работы

Данные дублируются на каждом диске в массиве. Если массив состоит из
двух дисков, данные записываются одновременно на оба, создавая полную
копию.

Назначение

RAID 1 предоставляет высокий уровень защиты данных. Если один из дисков
выйдет из строя, данные будут доступны на другом диске.

Примеры применения

RAID 1 часто используется для систем, требующих высокой надежности,
таких как файловые серверы и системы резервного копирования.

RAID 5

Алгоритм работы

Данные и контрольные суммы (паритетные данные) распределяются по всем
дискам в массиве. Для восстановления данных при отказе диска
используется информация о паритете, что позволяет обеспечивать
резервирование без полного дублирования данных.

Назначение

RAID 5 обеспечивает хороший баланс между производительностью и уровнем
защиты данных. Выдерживает отказ одного диска без потери данных.

Примеры применения

RAID 5 подходит для использования в серверах, приложениях с большими
объемами данных и системах, где важны как производительность, так и
резервирование (например, базы данных).

RAID 6

Алгоритм работы

Подобно RAID 5, но с дополнительным уровнем защиты. RAID 6 использует
две контрольные суммы (двойной паритет), что позволяет ему выдерживать
отказ двух дисков одновременно.

Назначение

RAID 6 обеспечивает высокий уровень защиты данных и подходит для систем,
где критически важна надежность.

Примеры применения

RAID 6 часто используется в крупных системах хранения данных, где риск
потери данных неприемлем, например, в облачных хранилищах, центрах
обработки данных и серверных кластерах.
\end{frame}

\begin{frame}{6. Выводы}
\phantomsection\label{ux432ux44bux432ux43eux434ux44b}
Этот процесс позволил мне получить практику в работе с RAID-массивами и
утилитами для его управления, а также познакомиться с основами настройки
и восстановления массивов.
\end{frame}




\end{document}
